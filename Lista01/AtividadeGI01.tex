\documentclass[a4paper,10pt]{article}

\usepackage[brazilian]{babel}
\usepackage[utf8]{inputenc}
\usepackage{verbatim}
\usepackage{graphicx}
\usepackage{cite}

\title{Roteiro para Atividade Prática}	
\author{Iann Carvalho Barbosa}
\date{10/06/2017}

\begin{document}
\maketitle

\tableofcontents
\newpage

\section{Questão 1}
\subsection{Pescador}
\subsubsection{Pesca de linha}
A pesca com linha e anzol, parecendo simples, continua a ser uma das principais formas de capturar peixe. Pelo fato do material ser de fácil aquisição, é o principal método de pesca de subsistência em rios, lagos ou junto à costa. No entanto, várias pescarias industrializadas usam este método, quer com a chamada linha-de-mão, em que cada pescador segura na mão uma linha na extremidade da qual se colocam várias linhas secundárias cada uma com o seu anzol, até aos palangres de vários quilómetros de comprimento com que se pescam os atuns de profundidade.

Ainda muito praticado mas com menos adeptos é a pesca com mosca ou fly fishing em Inglês.

A pesca de anzol é ainda um esporte muito praticado no mundo.

\subsubsection{Pesca de cerco}

Algumas variantes da rede de emalhar deram origem às redes de cerco: a rede é colocada em volta de um cardume e o cabo do fundo pode ser puxado até formar um saco onde todo o peixe fica aprisionado. Esta forma de pescar é utilizada tanto em nível artesanal - na região norte de Moçambique estas redes são fechadas por 4-5 mergulhadores, em águas baixas - como em nível industrial, por exemplo, para algumas espécies de atum que formam cardumes à superfície do mar.

\subsubsection{Pesca com armadilhas}
As armadilhas de diversos tipos são também métodos de pesca muito populares desde tempos imemoriais. Na região Indo-Pacífica, quer dizer nas zonas tropicais e subtropicais dos oceanos Índico e Pacífico, os pescadores locais constroem gaiolas em forma de V com ripas de bambu ou de folhas de palmeira, colocam-nas perto de rochas ou recifes de coral e conseguem capturar peixes de grande valor comercial. Em Portugal existe uma pesca tradicional para cefalópodes (principalmente polvo) com alcatruzes (que são recipientes de barro, normalmente presos em número variável a linhas suspensas na água) ou "covos" que são gaiolas fabricadas de arame ou fibras vegetais. Os "covos" são bastante utilizados na costa norte portuguesa (Matosinhos, Labruge, Vila Chã, Mindelo, Vila do Conde e outras). Estas artes de pesca, como se designam os instrumentos utilizados diretamente na captura de peixe e outros animais aquáticos, pertencem ao grupo das chamadas artes passivas, uma vez que é o próprio animal que as procura, normalmente como refúgio, ficando nelas aprisionado.

Ao nível industrial, há pescarias que utilizam gaiolas, construídas em plástico ou rede segura numa armação metálica, que podem ser colocadas em grandes números e em qualquer profundidade, presas a um cabo. Estas gaiolas provocam um problema semelhante ao das redes de emalhar derivantes, pois podem perder-se e continuar a matar peixes ou outros organismos, sem nenhum benefício, nem para o homem, nem para os próprios recursos pesqueiros.

\subsection{Piloto de fórmula 1}
\subsubsection {As cores dos carros}
As cores tradicionais dos carros no início da Fórmula 1 eram: o verde para as equipes inglesas, o vermelho para as italianas, o azul para as francesas e o branco alemão. Nessa época, a F-1 era essencialmente um esporte, e o mercantilismo ainda não tinha tomado conta. As equipes eram mantidas com ajuda das empresas de petróleo e fabricantes de pneus. Essa obrigação durou até 1968.
\subsubsection{A primeira mulher}
No ano de 1958, pela primeira vez a Fórmula 1 teve uma mulher piloto alinhando no grid. Foi a italiana Maria Teresa de Filippis.[1] Ela tentou se classificar para 5 grandes prêmios, quatro deles pela equipe Maserati e um pela Porsche. Classificou-se para três deles. Sua melhor atuação foi em sua segunda corrida, na Bélgica em 1958, quando largou na 15ª colocação e terminou na 10ª.
\subsubsection{Os melhores pilotos de Fórmula 1 de todos os tempos}
\begin{itemize}
\item Autosport
\begin{enumerate}
\item Brasil - Ayrton Senna
\item Alemanha - Michael Schumacher
\item Argentina - Juan Manuel Fangio
\item França - Alain Prost
\item Reino Unido - Jim Clark
\item Reino Unido - Jackie Stewart
\item Áustria - Niki Lauda
\item Reino Unido - Stirling Moss
\item Espanha - Fernando Alonso
\item Canadá - Gilles Villeneuve
\item Reino Unido - Nigel Mansell
\item Brasil - Emerson Fittipaldi
\item Brasil - Nelson Piquet
\item Áustria - Jochen Rindt
\item Finlândia - Mika Hakkinen
\item Itália - Alberto Ascari
\item Reino Unido - Lewis Hamilton
\item Austrália - Jack Brabham
\item Suécia - Ronnie Peterson
\item Estados Unidos - Mario Andretti
\item Reino Unido - Graham Hill
\item Finlândia - Kimi Raikkonen
\item Reino Unido - John Surtees
\item Reino Unido - James Hunt
\item Finlândia - Keke Rosberg
\item Alemanha - Sebastian Vettel
\item Argentina - Jose Froilan Gonzalez
\item Estados Unidos - Dan Gurney
\item França - François Cevert
\item Reino Unido - Jenson Button
\item Austrália - Alan Jones
\item Itália - Giuseppe Farina
\item Estados Unidos - Phil Hill
\item Argentina Carlos - Reutemann
\item Alemanha - Stefan Bellof
\item Suíça - Clay Regazzoni
\item Áustria - Gerhard Berger
\item França - Jean Behra
\item Itália - Riccardo Patrese
\item Bélgica - Jacky Ickx
\end{enumerate}
\item The Times
\begin{enumerate}
\item Reino Unido - Jim Clark
\item Brasil - Ayrton Senna
\item Alemanha - Michael Schumacher
\item França - Alain Prost
\item Reino Unido - Jackie Stewart
\item Argentina - Juan Manuel Fangio
\item Reino Unido - Stirling Moss
\item Espanha - Fernando Alonso
\item Reino Unido - Nigel Mansell
\item Finlândia - Mika Hakkinen
\item Itália - Alberto Ascari
\item Reino Unido - Graham Hill
\item Finlândia - Kimi Raikkonen
\item Áustria - Niki Lauda
\item Brasil - Nelson Piquet
\item Reino Unido - Jenson Button
\item Reino Unido - James Hunt
\item Áustria - Jochen Rindt
\item Canadá - Gilles Villeneuve
\item Austrália - Jack Brabham
\item Reino Unido - Lewis Hamilton
\item Brasil - Emerson Fittipaldi
\item Austrália - Alan Jones
\item Finlândia - Keke Rosberg
\item Canadá - Jacques Villeneuve
\item Reino Unido - Mike Hawthorn
\item Estados Unidos - Mario Andretti
\item Nova Zelândia - Bruce McLaren
\item Reino Unido - John Surtees
\item Colômbia - Juan Pablo Montoya
\item Reino Unido - Damon Hill
\item Nova Zelândia - Denny Hulme
\item Reino Unido - David Coulthard
\item França - Didier Pironi
\item Suécia - Ronnie Peterson
\item Brasil - Felipe Massa
\item África do Sul - Jody Scheckter
\item Brasil - Rubens Barrichello
\item Bélgica - Jacky Ickx
\item Argentina - Carlos Reutemann
\item Reino Unido - Tony Brooks
\item Estados Unidos - Phil Hill
\item Itália - Giuseppe Farina
\item Suíça - Jo Siffert
\item Itália - Lorenzo Bandini
\item Áustria - Gerhard Berger
\item Estados Unidos - Dan Gurney
\item Suíça - Clay Regazzoni
\item Reino Unido - Peter Collins
\item Itália - Michele Alboreto
\end{enumerate}

\end{itemize}


\subsection{Político}
\subsubsection{Quem é considerado político?}
É o indivíduo pertencente a um partido, que preocupa-se em obter aceitação da população para ascender a uma determinada posição. Participa ativamente de política partidária. Tem o poder de formar opinião pública. Num Estado, são os membros dos poderes executivo e legislativo, do governo federal, dos governos estaduais e municipais. Também pode-se considerar político alguém que manipule e influencie a opinião de um determinado grupo em favor de uma ideia. Também se pode considerar alguém que não sabendo fazer mais nada, se serve dos poderes que a política lhes dá e consequentemente o mando para "se governar".

\subsubsection{Quem não é considerado político?}
\begin{enumerate}
\item Membros do governo que sirvam meramente para trabalhos burocráticos, como assessores e consultores técnicos;
\item Membros concursados chamados de funcionários públicos, sejam eles do poder executivo, do poder legislativo, do poder judiciário ou militares, não são, geralmente, considerados políticos, embora estejam envolvidos nos processos de decisão do governo;
\item Cidadãos comuns com poder de voto não são exatamente considerados políticos, embora possam ser formadores de opinião pública.
\end{enumerate}

\subsubsection{Cargos políticos}
Alguns cargos políticos são:
\begin{enumerate}
\item Congressista
\item Governador
\item Deputado
\item Senador
\item Parlamentar
\item Ministro
\item Prefeito
\item Primeiro-ministro
\item Vereador
\end{enumerate}
Um indivíduo candidato à eleição para qualquer um desses cargos é, geralmente, definido como político.

\subsection{Turista}
\subsubsection{Categorias}
Ainda segundo a OMT, dependendo de uma pessoa estar em viagem para, de ou dentro de um certo país, as seguintes formas podem ser distinguidas:
\begin{enumerate}
\item Turismo receptivo - quando não-residentes são recebidos por um país de destino, do ponto de vista desse destino.
\item Turismo emissivo - quando residentes viajam a outro país, do ponto de vista do país de origem.
\item Turismo doméstico - quando residentes de dado país viajam dentro dos limites do mesmo.
\end{enumerate}
\subsubsection{Importância econômica}
O Turismo é a atividade do setor terciário que mais cresce no Brasil (dentre as espécies, significativamente, o turismo ecológico, o turismo de aventura e os cruzeiros marítimos) e no mundo, movimentando, direta ou indiretamente mais de 4 trilhões (2004) dólares americanos, criando também, direta ou indiretamente, 170 milhões de postos de trabalho, o que representa 1 de cada 9 empregos criados no mundo.

Tal ramo é de fundamental importância para o profissionalismo do setor turístico e necessário para a economia de diversos países com excelente potencial turístico, como o Brasil.

No Brasil, cidades médias e pequenas que são desprovidas de um próprio centro financeiro, precisam de meios para o crescimento de sua economia e de seu desenvolvimento. Alguns exemplos sobre esse caso são: Vitória, Guarujá, Ilha Bela, Ubatuba, Ouro Preto, Tiradentes, Paraty, Angra dos Reis, Armação dos Búzios, Cabo Frio, entre outras.

Grandes metrópoles globais também usam o turismo para sua fonte econômica, apesar de terem uma ampla economia de influência nacional ou internacional, como: São Paulo, Rio de Janeiro, Buenos Aires, Nova York, Los Angeles, Londres, Paris, Tóquio, entre outras. Muitas delas utilizam diversos tipos de turismo, como: de negócios, lazer, cultural, ecológico(mais aplicado em cidades menores com maior área rural, apesar de existirem reservas florestais em algumas metrópoles), etc.

Em outros países, entre desenvolvidos e subdesenvolvidos, ocorre o mesmo. Nos Estados Unidos da América, o estado do Havaí, além de ser uma ilha distante do continente, possui também pouca população, em comparação a outros estados, sendo assim, difícil de ter um maior maior crescimento na sua economia. Portanto, o estado teve de optar para o turismo, e hoje é um dos mais famosos pontos turísticos dos Estados Unidos, sendo conhecido por suas belas praias e resorts.

Atualmente, um dos locais que mais crescem com o turismo, é a cidade de Dubai, nos Emirados Árabes Unidos. Pela sua localização, próxima à regiões de conflitos étinicos e religiosos, a cidade teve de enfrentar muitos obstáculos para ser conhecida em diferentes partes do mundo. Conta com os mais exóticos e originais arranha-céus, sendo muitos deles, hotéis, tendo destaque para o Burj Al Arab, cartão postal da cidade, e para o Rose Tower, o hotel mais alto do mundo.
\subsubsection{Estudo do turismo}
O estudo do turismo é uma área de recente desenvolvimento dentre as ciências sociais aplicadas. No Brasil os primeiros cursos superiores na área sugiram no início da década de 1970, destacando-se aqueles criados na antiga Faculdade Anhembi Morumbi e na Universidade de São Paulo, e também o da Pontifícia Universidade Católica do Rio Grande do Sul (PUCRS).

Apesar de controverso, o termo turismologia têm sido utilizado para designar essa área de estudos, sendo turismólogo o termo utilizado designar o estudioso da área.

\subsection{Astronauta}
\subsubsection{Terminologia}
De certa forma, "astronauta", "cosmonauta", "taikonauta" e "espaçonauta" são sinônimos do termo "viajantes espaciais". Na maior parte das vezes, "cosmonauta" e "astronauta" são sinônimos em todas as línguas, e o uso da escolha é frequentemente ditado por razões políticas, sendo que ambos os termos ficaram consagrados durante a corrida espacial da década de 1960, disputada entre os Estados Unidos e a ex-União Soviética.
\subsubsection{Treinamento}
No início do programa espacial, os pré-requisitos para uma pessoa tornar-se astronauta da NASA seria ter formação em engenharia e pilotagem de aviões militares a jato, embora nem John Glenn e nem Scott Carpenter (ambos do Mercury Seven) tivessem qualquer grau universitário de engenharia ou de qualquer outra disciplina na altura da seleção. A seleção foi inicialmente limitada aos pilotos militares. Os primeiros astronautas (da América e da Rússia) tendiam a ser pilotos de aeronaves, e foram muitas vezes pilotos de testes.

Uma vez selecionados, os astronautas da NASA passam por um treinamento de 20 meses em uma variedade de áreas, incluindo testes de atividade extraveicular em uma instalação como a Neutral Buoyancy Laboratory da própria NASA. O astronauta em treinamento também pode experimentar curtos períodos de microgravidade.

O astronauta tem a obrigação de acumular um certo número de horas de voo em aviões de jato de capacidade elevada antes da de(s)colagem. Isto é feito principalmente em veículos como o T-38 Talon fora de Ellington Field, devido à sua proximidade com o Lyndon B. Johnson Space Center. A maioria dos voos das aeronaves são feitas fora da Edwards Air Force Base.
\subsubsection{Marcos de viagens espaciais}
O primeiro homem a atingir o espaço foi o russo Iuri Gagarin, lançado em 12 de abril de 1961, a bordo da Vostok 1, espaçonave que pesava 4725 kg. Quando Iuri viu a Terra do espaço, pronunciou as palavras no controle terrestre: "A Terra é azul. Como é maravilhosa. É surpreendente." Valentina Tereshkova foi a primeira cosmonauta da história e a primeira mulher a ir ao espaço, em 16 de junho de 1963.

Alan Shepard tornou-se o primeiro americano e a segunda pessoa a viajar no espaço, em 5 de maio de 1961. A primeira mulher americana no espaço foi Sally Ride, durante a missão STS-7 em 18 de Junho de 1983, a bordo da Challenger (ônibus espacial). A primeira missão para a órbita da lua foi a Apollo 8, que incluiu William Anders, nascido em Hong Kong, fazendo dele o primeiro astronauta asiático, em 1968. Em 15 de Outubro de 2003, o primeiro astronauta chinês foi Yang Liwei, na missão Shenzhou 5.

A União Soviética, através do Programa Intercosmos, permitiu que pessoas de outros países socialistas voassem em suas missões. Uma dessas pessoas foi o checoslovaco Vladimir Remek, que se tornou o primeiro europeu no domínio espacial soviético em 1978, pelo foguete russo Soyuz. Em 1980, o cubano Arnaldo Tamayo Méndez tornou-se a primeira pessoa descendente de africanos a voar no espaço (o primeiro africano no espaço foi Patrick Baudry, em 1985). Em 1988, Abdul Ahad Mohmand tornou-se o primeiro afegão a partir fora da terra, retornando nove dias depois a bordo da estação espacial Mir.

O primeiro lusófono a partir para uma tripulação espacial foi o brasileiro Marcos Pontes, a bordo da "Missão Centenário". Em 30 de março de 2006, partiu em direção à Estação Espacial Internacional (ISS) a bordo da nave russa Soyuz TMA-8, com oito experimentos científicos brasileiros para execução em ambiente de microgravidade, retornando no dia 8 de abril a bordo da nave Soyuz TMA-7. Além de ter sido o primeiro lusófono, foi o primeiro homem do Hemisfério Sul a ir para o espaço. 

\subsection{Eleitor}
\subsubsection{História}
Em 1875 foram criados pela Lei do Terço dois documentos que comprovavam a situação eleitoral do cidadão. Para os votantes (primeiro grau) foi criado o título de qualificação, enquanto que para os eleitores (segundo grau) criou-se o diploma de eleitor geral.

Em 1881, A Lei Saraiva promoveu uma ampla reforma eleitoral e, entre outras alterações, determinou o fim da eleição em dois graus e instituiu o título de eleitor obrigatório.

Em 1932, o titulo passou a ter a foto do eleitor e em 1956 o retrato tornou-se obrigatório. Em 1986, foi definido o novo modelo, sem o retrato, que é utilizado até os dias de hoje.
\subsubsection{Critérios para a primeira emissão do título de eleitor}
Para se obter o título eleitoral, o cidadão deve comparecer ao cartório eleitoral que atende do município (ou bairro) onde mora, de posse dos seguintes documentos:
\begin{enumerate}
\item Cédula de identidade.
\item Comprovante de quitação do serviço militar (no caso de cidadãos do sexo masculino com idade entre 18 e 45 anos).
\item Comprovante de residência (conta de luz, telefone, água etc.).
\item Pode-se tirar o título a partir do ano em que o cidadão completar 16 anos e com essa idade basta levar comprovante de residência e RG, originais e cópias.
\end{enumerate}
\subsubsection{Na falta do título do eleitor}
O documento é obrigatório a todos os cidadãos com mais de 18 anos e facultativo para aqueles entre 16 e 18, ou com mais de 70. O título é também opcional para cidadãos não alfabetizados.

O eleitor que não possui o título, ou se este se encontra cancelado, não pode solicitar a emissão de passaporte ou do cartão do CPF, bem como inscrever-se em concurso público, renovar a matrícula em estabelecimentos oficiais de ensino e obter empréstimos em caixas econômicas federais e estaduais.

Até as eleições de 2008, a apresentação do título eleitoral não era necessária para votar. Para eleitores que estivessem regularmente inscritos na Justiça Eleitoral, bastava apresentar um documento de identidade com foto em sua seção correspondente.

A eliminação da foto do título de eleitor, em 1996, facilitou muito as fraudes de identidade (em que uma pessoa vota no lugar de outra). Para reduzir esse risco, foi aprovada legislação em 2007 (Lei 504/97, artigo 91-A) tornando obrigatória a apresentação de um documento de identidade oficial com foto (como a cédula de identidade ou o passaporte), além do título de eleitor, a partir das eleições de 2010. Com a aproximação dessas eleições, entretanto, surgiu o temor de que milhões de eleitores, especialmente dentre os mais pobres, poderiam ser impedidos de votar por falta de um dos documentos. O Partido dos Trabalhadores então questionou a constitucionalidade dessa lei na justiça, com o argumento de que todo cidadão devidamente registrado como eleitor tem o direito de votar se conseguir comprovar sua identidade. Em uma decisão de última hora (30 de setembro de 2010), o Supremo Tribunal Federal acatou o argumento e determinou que para votar em 2010 bastaria apresentar o documento de identidade com foto. Com essa decisão, o título de eleitor brasileiro ficou privado de sua função principal.

\section{Questão 2}
De acordo com o dicionário, a definição de inteligência é: "Inteligência é um conjunto que forma todas as características intelectuais de um indivíduo, ou seja, a faculdade de conhecer, compreender, raciocinar, pensar e interpretar." e a de ética seria: "Parte da filosofia responsável pela investigação dos princípios que motivam, distorcem, disciplinam ou orientam o comportamento humano, refletindo esp. a respeito da essência das normas, valores, prescrições e exortações presentes em qualquer realidade social."

Diante disso, como o saber é o conhecimento usado de forma inteligente e ética, um bom exemplo de como isso pode acontecer é o que eu sonho para o meu futuro profissional em ciência da computação, porque pretendo usar o conhecimento provindo da universidade para tornar o mundo em um lugar melhor, criando softwares para pessoas com deficiências físicas ou pouco favorecidas financeramente, por exemplo.

\section{Questão 3}
\subsection{Definição de dados}
Os dados são os fatos em sua forma primária que por si só não transmite nenhuma mensagem, ou seja, são os fatos recebidos por seus cinco sentidos, sejam eles através da visão: imagem, texto, cor ou qualquer outro estímulo sensorial. Por exemplo, um dado é uma placa de trânsito, uma música, um livro etc.
\subsection{Definição de informação}
São dados analisados e interpretados sob determinada óptica, vale salientar que pode ser muito relativa, pois os seres humanos podem interpretar de diferentes formas o mesmo estímulo sensorial. Por exemplo, quando descoberto que uma viagem de Campina Grande para João Pessoa é de 134 km, dois indivíduos diferentes podem processar esse dado e um dizer que é longe e o outro dizer que é perto.
\subsection{Definição de conhecimento}
O conhecimento é a aplicação da informação absorvida pelo indivíduo construida pela sua respectiva experiência, aprendizado e/ou comunicação, ou seja, diante das informações relativas pode-se criar uma solução para problemas. Por exemplo,  no dia dos namorados, um rapaz pensa e conclui que é um dia especial, que a sua namorada é uma menina muito boa e que ela gosta de chocolate, então ele decide comprar uma caixa de chocolate e flores para presentea-la.

\section{Questão 4}
Na minha vida pessoal tive que tomar uma decisão recentemente se trancaria uma cadeira ou não por decorrencia de problemas pessoais.'
\begin{enumerate}
\item Trancar a cadeira.
\item Se eu não trancasse, ficaria muito pesado e difícil e se eu trancasse, demoraria mais para terminar o curso.
\item Se eu ficar com todas as cadeiras, posso perder uma ou talvez mais e baixar meu CRA, mas se eu ficar com todas as cadeiras posso terminar o curso mais cedo.
\item Preciso ter certeza de que é melhor não correr o risco de perder cadeiras e deixar o curso mais longo.
\item É melhor ter certeza de que pagarei todas as cadeiras e que, mesmo diante de problemas pessoais, pois além disso, posso aumentar meu CRA, visando novos projetos.
\item Preciso me comunicar, através do controle acadêmico, com o coordenador para pedir trancamento da cadeira.
\item O período ficou mais leve e consigo dar conta da minha vida acadêmica e pessoal.
  \end{enumerate}
  \end{document}  